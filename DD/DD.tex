\documentclass[titlepage]{article}
\usepackage[T1]{fontenc}
\usepackage[utf8x]{inputenc}
\usepackage[english]{babel}
\author{Luca Massini \and Daniele Nicolò}

\title{Design Document
\\  Safestreets}

\date{release date to be defined}
\begin{document}
\maketitle
\newpage 
\tableofcontents
\newpage
\section{Introduction}
\subsection{Purpose}
	This document is necessary to describe the architecture of the system from several points of view. An architecture can be expressed in terms of the system components and their interactions with each other.\\
	An architecture can be thought as a starting point for possible future changes,according to the needs of the stakeholders.\\
	Finally the requirements mentioned in the RASD document will be explained here from an architectural perspective, that can be seen with component, deployment and runtime views. 
\subsection{Scope}
Safestreets is a service that allows private users to inform authorities about parking and traffic violations. A user must take a picture of the violation and describe it and the place where it occurred. An image recognition algorithm is run on the sent picture. A user has also the possibility to see the safety of the streets and the parking areas and to check the most reported streets and vehicles. \\
The system offers also the possibility to receive suggestions in order to improve the streets safety. This feature is available only for municipality accounts. These  suggestions are generated by the system using an algorithm that retrieves the information from the users' reports and also from the data given by the municipality.
\subsection{Definitions,Acronyms and Abbreviations}
\subsubsection{Definitions}
TODO
\subsubsection{Acronyms}
TODO
\subsubsection{Abbreviations}
TODO
\subsection{Revision History}
TODO
\subsection{Reference Documents}
TODO
\subsection{Document Structure}
The following sections are structured as below:
\begin{itemize}
	\item \textbf{Architectural Design: }In this section there will be the description of the system from the architectural point of view. This means to show the components and their interactions with each other and to explain the design patterns choices and the architectural styles. 
	\item \textbf{User Interface Design:} In this section there is an explanation,in terms of user experience (UX), of the user interfaces already showed in the RASD mockups.
	\item \textbf{Requirements Traceability:} Here we describe how the requirements already explained in the RASD match with the design choices done in this document.
	\item \textbf{Implementation, Integration and Test plan:} Here there is the description of how we will implement all the components, how we will integrate them together and finally how we will test both the single components and also the integrated system.
	\item \textbf{Effort spent:} Here there is the division of the work hours of each member of the group and the description of the tasks completed and related time spent.
\end{itemize}
	
\end{document}