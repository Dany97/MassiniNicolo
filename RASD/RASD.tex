\documentclass[titlepage]{article}
\usepackage[T1]{fontenc}
\usepackage[utf8]{inputenc}
\usepackage[italian]{babel}


\author{Luca Massini \and Daniele Nicolò}

\title{Requirement analysis and specification document
\\ Safestreets}

\date{release date to be defined}
\begin{document}
\maketitle
\newpage 
\tableofcontents
\newpage

\section{Introduction}
The purpose of this document is to represent the Requirement Analysis and Specification Document (RASD).
This document shows what are the goals and the requirements of the software.
It has to represent how the application can be useful for the users that will use it and why they are fundamental to improve the quality of the service offered. Secondly, this document can be also used as a support for the testing of the system, for the verification activities and also the validation ones. The RASD can be also used to guide the changes in a already existing system.
\subsection{Purpose}

\subsubsection{Descripton of the given problem:}
SafeStreets is a software useful to help people to be safer when they are on the street.\\
The users can send to the municipality pictures of violation occurring in public streets: the reporting can concern violation on the road, in a parking and so on.\\
The software allows the users to send detailed information about the violation, such as the hour, the date, the type of violation and the position (they can be captured with GPS).\\
Furthermore, the service can provide, both to the user and to the municipality,information about the streets in which the user is around,
such as the number of violations per street and consequently the level of danger of the street.\\
In addition, the user can have a different service based on the category to which he/she belongs.\\
The user and the municipality can also find on the application the most “dangerous” vehicles, that are the ones with the highest number of reports from the users.\\
The service must be different in base of the type of user, such  authorities, motorists, motorcyclists, bikers, pedestrians, disabled people, etc, so it must be easy to use.\\
Finally, the users must be able to receive recommendations from the system to avoid using streets, parking lots that are risky in general or at a specific hour or date.\\

\subsubsection{Goals:}
\begin{itemize}
\item $[goal 1]$:  The service allows the users to report a violation to the municipality.
\begin{itemize}

	\item $[goal 1.1]$: The user can send a picture of the 			violation.
	
	\item $[goal 1.2]$: The user can specify the date and the 						hour when the violation has happened 							and the type of violation.
	
	\item $[goal 1.3]$: The position of the violation can be 			                taken in an automatic way (by using 							the GPS) or it can be written by the 							user. It's a user choice.
	
	\item $[goal 1.4]$: The date and also the hour can be           		  taken in an automatic way or they can be written          		  manually by the user. It's a user choice.
	
	\item $[goal 1.3 ]$: The municipality must be able to     	receive the reports  \\
	
\end{itemize}

\item $[goal 2]$: The service allows the users to have detailed information about the violations and the street safety
      \begin{itemize}
      	\item $[goal 2.1]$: The users can know which are the 			most reported streets, areas or parking.
      	
      	\item $goal 2.2]$: The users can see which are the 				vehicles that commit the highest number
		Of traffic violations.
		
		\item $[goal 2.3]$: The users, in base of his/her 				position, can be recommended to use the safest 					streets/areas, etc or avoid the dangerous ones, that 			are highlighted in different ways on the map.\\

      \end{itemize}
\item $[goal 3]$: The service offers different types of 						  interfaces and accessibility in according 					  to the type of user (biker, pedestrian, 						  rider, driver, disabled person).
	\begin{itemize}
	\item $[goal 3.1]$: The user that want to park his/her  							car/motorbike can receive a                           						suggestion of where to do it 
						based on the safest streets around 								him/her.\\
	\end{itemize}

\item $[goal 4]$: The service allows the user to be 							  registered in the system with a username 				  		  and a password.\\

\item $[goal 5]$: A user can specify the category to which 						  he/she belongs (motorcyclist, biker, 							  pedestrian, car driver, disabled person).In 				  order to improve the service quality.\\

\item $[goal 6]$: A driver or a motorcyclist can add to his 					  personal profile the license plate of his 					  vehicle.\\

\item $[goal 7]$: Safestreet can send to the municipality 						  suggestions of what to do in order to 						  improve the safety on the streets for every 
				  type of Safestreets user.
	\begin{itemize}
	\item $[goal 7.1]$: The municipality must receive every 
					    suggestion sent by Safestreets.\\
					   
	
	\end{itemize}

\end{itemize}

\subsubsection{Scope:}
In this section is introduced the so called "world and machine fenomena".\\
We distinguish the world and the machine. The world is related to every phenomena that take place outside of the region of events that the system to be developed (called machine) is able to control or eventually observe. Instead, the machine is related to phenomena which happen inside the machine and so they're completely controllable. Some of the totality of the phenomena can be observed by the machine but they're controlled by the world.  and viceversa. These are called shared phenomena. In this section there will be described only the world phenomena and the shared one.The machine phenomena will be described later on in the next chapters.

\paragraph{World phenomena: }
\subparagraph{From the user point of view: }
\begin{itemize}	
	\item The user sees one or more violations.
	\item The user wants to report a violation to the 					  municipality.
	\item The user wants to know the area/streets where the 
	     major violations occur.
	\item The user wants to know where is safer to park his/			  her car/bike/motorcycle.
	\item The user wants to know what are the vehicles that       		  commit the major number of violations.

\end{itemize}
\subparagraph{From the municipality point of view: }

\begin{itemize}
	\item The municipality wants to know what to do to 				improve the safety of the streets.
	\item The municipality wants to discover what're the most
	dangerous zones of the city.\\
\end{itemize}
 
\paragraph{Shared phenomena: }
\subparagraph{Shared phenomena controlled by the world and observed by the machine: }
\begin{itemize}
 	\item The user sends to the system a picture of a            	violation and all the other data relating to it.
 	\item The user sends some data request to the system.
	\item The municipality sends a data request to the system 		  in order to receive suggestion to improve the 				  safety 	of the streets.
	\item The municipality sends a data request to know what 			  are the vehicles that have commited the major 				  number of accidents.
\end{itemize}
\subparagraph{Shared phenomena controlled by the machine and  			observed by the world: }
\begin{itemize}
	\item the user receive data related to what he/she has     	asked
\end{itemize}
\end{document}
